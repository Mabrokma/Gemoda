\documentclass[]{article}

\begin{document}

%% Adjustment of edges in the currently induced graph.

After cliques are found at the current threshold (maximum) in the
current graph, we must reduce the related values in the graph so that
these cliques or some portion of them (with reduced length or support)
are not found again later.  To accomplish this, we reduce the weight of
each edge to the largest weight at which it would participate in some
new clique.  To participate in a new clique means that some node that
was not already part of a clique that included the two nodes connected
to the edge to be reduced would end up being part of the clique (fully
connected to each of the other nodes in the clique) at some lower
threshold, such that its edges to each of the other nodes would also be
induced.  The point at which all such edges would be induced initially
is when the threshold reaches the minimum among the weights of all such
edges, which will be the first point at which all the edges will be
induced.

In our current graph, we may use $w(a, b)$ to indicate the weight of the
edge between nodes $a$ and $b$.  We will also consider $T$ to be the set
of pairs of nodes (cells in the adjacency matrix) that have an edge
between them in the induced graph at the current threshold.  That is,
$(a, b) \in T \Longleftrightarrow w(a, b) >= t$.

Start with an edge between two nodes that is induced at the current
threshold: $(a, b) \in T$.  For any third node $x$, $(a, b, x)$ will
become a clique whenever the weights of each of the edges between them
meet the threshold, and thus first when the threshold reaches the
minimum of the weights of each edge: $\min \{ w(a, b), w(a, x), w(a, y)
\}$.  Thus to find the largest threshold at which a clique will exist
among nodes $a$, $b$, and some other node $x$, we calculate:

\begin{eqnarray*}
\max_x \left\{ \min \{ w(a, b), w(a, x), w(b, x) \} \right\}
 &=& \max_x \left\{ \min \{ t, w(a, x), w(b, x) \} \right\} \\
 &=& \max_x \left\{ \min \{ w(a, x), w(b, x) \} \right\}
\end{eqnarray*}
.

Of course, for some nodes $x$, the nodes $(a, b, x)$ may already form a
clique at the current threshold, in which case we want to disqualify
that node $x$ from the above formula so that we do not simply find that
same clique again.  To do so, we need to require:

\begin{displaymath}
(a, x) \not\in T \vee (b, x) \not\in T
\end{displaymath}
.

Thus, we must skip nodes $x_i$ for which $(a, x_i) \in T \wedge (b,
x_i) \in T$, because for each such $x_i$, we have already found the
clique $(a, b, x_i)$ at the current threshold.  However, these 3-cliques
can also combine to form new cliques.  In particular, if we have an
$x_i$ and $x_j$ that are each already in 3-cliques with $(a, b)$, we
will next form the 4-clique containing all four nodes at the minimum
among all weights between them:

\begin{eqnarray*}
\min \{ w(a, b), w(a, x_i), w(b, x_i), w(a, x_j), w(b, x_j), w(x_i, x_j)
\}
 &=& \min \{ t, t, t, t, t, w(x_i, x_j) \} \\
 &=& w(x_i, x_j)
\end{eqnarray*}

The simplification is due to the fact that each of $x_i$ and $x_j$
already forms a clique with $(a, b)$ at the current threshold, and the
weight of any edge can never be greater than the current threshold.  In
particular, among all such nodes that already participate in cliques
with $(a, b)$, we will find the next 4-clique combining two of them
along with $(a, b)$ to be:

\begin{eqnarray}
\max_{i, j} \left\{ w(x_i, x_j) \right\}, i \neq j, \forall i:
(a, x_i) \in T \wedge (b, x_i) \in T
\end{eqnarray}
.

Again, it may be the case that some of these pairs $x_i$, $x_j$ may
already be in 4-cliques of $(a, b, x_i, x_j)$ at the current threshold,
in which case we want to avoid finding them again.  To disqualify such
pairs, we must simply add the requirement $(x_i, x_j) \not\in T$ to the
above formula.

At this point, we might consider combinations of larger cliques
containing $(a, b)$ in the same way we considered combining pairs of
3-cliques.  However, for an n-clique containing $(a, b)$ to combine with
some additional node $y$, the additional node must not already be
connected to all n nodes in the clique in the induced graph at the
current threshold, or else $y$ would already be included in the clique
at the current threshold and would thus be disqualified.  If $y$ does
not already connect to both $a$ and $b$ in the current induced graph,
then it will already be subsumed in the first calculation of combining
$(a, b)$ with each other node not already connected to both.  If $y$ is
already found in 3-clique $(a, b, y)$, then it must not be connected to
at least one $x_i$ within the n-clique in order to be considered a new
extension that will only happen at some lower threshold than the current
value.  In that case, the combination of $x_i$ and $y$ will already be
considered in the second calculation, combining pairs of nodes that
already participate in cliques with $(a, b)$, but not with each other.
Therefore, any higher level of combination of cliques would already be
manifested in the two low-level combinations described, and simply the
maximum among the two calculations is guaranteed to be the largest value
for the threshold at which $(a, b)$ will next participate in some new
clique.

%% Adjustment of edges downstream of those in the currently induced
%% graph.

After reducing each edge that is already induced (participates in one or
more cliques) at the current threshold, we must also reduce edges that
are ``downstream'' from those induced.  By this, we mean those pairs of
nodes that sequentially follow nodes that are connected at the current
induced subgraph.  When a clique is found with nodes $a$ and $b$ at the
current threshold $t$, we do not want to find the clique $(a + 1, b +
1)$ at threshold $t - 1$, since that is only a shorter piece of the
pattern already found with the clique $(a, b)$.  Therefore, for each
edge $(a, b)$ in the current induced subgraph, we must reduced each of
the edges $(a + k, b + k)$ sequentially following the connected nodes,
for $k$ going from $0$ to $t - 1$.  For each of these, we must again do
calculations similar to those done above for pairs of nodes connected in
the induced subgraph.

First of all, for simplicity, we will consider those cases where $(a +
k, b + k) \not\in T \forall k \in \{ 1, \ldots, t-1 \}$.  Unlike before,
we cannot simplify using the fact that $w(a, b) = t$ since $w(a + k, b +
k) \neq t$.  For the initial case of $(a + k, b + k)$ combining with
some new node $x$, we have:

\begin{eqnarray}
\max_x \left\{ \min \{ w(a + k, b + k), w(a + k, x), w(b + k, x) \} \right\}
\end{eqnarray}

again with the requirement to disqualify redundant pieces of patterns
that have already been found as cliques at the current threshold:

\begin{displaymath}
(a, x - k) \not\in T \vee (b, x - k) \not\in T
\end{displaymath}
.

Again, we are skipping those nodes $x_i$ for which $(a, x_i - k) \in T
\wedge (b, x_i - k) \in T$, since for each of these we have already
found the clique $(a, b, x_i - k)$ at the current threshold, so $(a + k,
b + k, x_i)$ will simply be a piece of that existing clique rather than
a new clique.  But again, we must also consider new combinations among
these nodes $x_i$ to form new 4-cliques with $(a + k, b + k)$:

\begin{displaymath}
\min \{ w(a + k, b + k), w(a + k, x_i), w(b + k, x_i),
        w(a + k, x_j), w(b + k, x_j), w(x_i, x_j) \}
\end{displaymath}

with the condition $(x_i - k, x_j - k) \not\in T$.

In particular, among all such nodes that already participate in cliques
with $(a, b)$, we will find the next 4-clique combining two of them
along with $(a + k, b + k)$ to be:

\begin{equation}
\max_{i, j} \left\{
\min \{ w(a + k, b + k), w(a + k, x_i), w(b + k, x_i),
        w(a + k, x_j), w(b + k, x_j), w(x_i, x_j) \}
\right\}, i \neq j, \forall i:
(a, x_i - k) \in T \wedge (b, x_i - k) \in T
\end{equation}

again with the requirement $(x_i - k, x_j - k) \not\in T$ to force us to
only consider new 4-cliques, rather than simply finding pieces of
patterns that have already been found due to cliques at the current
threshold.  Again, the combination of $(a + k, b + k)$ with other nodes
not already ``downstream'' of a clique with $(a, b)$ along with the
combination of those nodes that are ``downstream'' of such cliques, but
not already found in cliques with each other, guarantees that we will
find the next threshold where $(a + k, b + k)$ will participate in a
clique that is not part of one that has already been found, because any
larger cliques that are combined will also include these smaller
combinations.

%% Adjustment of edges downstream of one or more in the currently induced
%% graph.

The remaining case to be considered is when for some value $k'$, not
only is $(a, b)$ induced at the current threshold, but also $(a + k', b
+ k')$ is induced.  At this level, there may be (and in fact will be)
new nodes $x_i$ which are in cliques with $(a + k', b + k')$, while
these did not form cliques $(a, b, x_i - k')$ upstream.  Thus whenever
we have an induced edge $(a + k', b + k')$ as well the original induced
edge $(a, b)$, we must add some new nodes to the disqualified $x_i$
nodes that already participate in a clique found upstream of the current
$(a + k, b + k)$.  That is, our disqualified nodes $x_i$ when
considering the reduction of $(a + k, b + k)$ will contain all nodes
that form a clique $(a + k', b + k', x_i - k + k')$ at some point
upstream ($k' <= k$) of the current point.  Once again, we end up with
the same calculations for any $(a + k, b + k)$, except that we must
include new nodes $x_i$ each time one participates in an upstream clique
with some $(a + k', b + k')$, and we must ignore those combinations of
$(a + k, b + k, x_i, x_j)$ when these are all downstream of some clique
that already exists in the current induced graph: $(a + k', b + k',
x_i - k + k', x_j - k + k')$.

In general, our process is actually one simple idea.  At each edge $(a,
b)$ in the current induced graph, we must reduce its weight, as well as
those of all subsequent nodes $(a + k, b + k)$ which consist of shorter
pieces of those patterns and cliques already found, to the next
(largest) threshold at which some node will be added to an existing
clique containing the pair of nodes whose edge is being reduced.  We
first consider single nodes which will combine with these two nodes to
form a new 3-clique, while excluding those third nodes which already
form a clique with these two nodes or at some point upstream.  Then
pairs of those excluded nodes that did not already form a clique along
with the two original nodes at this point or any point upstream can also
be combined to form new 4-cliques.  The maximum value at which either of
these new combinations occur will be the next threshold at which the
edge under consideration will participate in a new clique, and thus is
the value to which its weight should be reduced.

This reduction calculation ensures that only new (not redundant, meaning
not simply being part of some larger pattern with either greater support
or greater length) patterns will be found, while also allowing all such
non-redundant patterns to be found.

\end{document}
